\newpage
\appendix
%
\section{Appendix}\label{sec:appendix}
%
\subsection{Your Report}
Together with the report, submit your original measurement data as a neat \texttt{.txt}-file with appropriate comments so that the data can be assigned to the respective measurements. The structure of the protocol should be based on the following points:
\begin{itemize}
	\item Theory section
	\item Execution and evaluation
	\begin{itemize}
		\item What is to be measured and why?
		\item What measurement results are expected?
		\item How is the measurement carried out?
		\item What was actually measured? How large are the measurement errors?
		\item What can be learned from the measurement result? Were the expectations met?
		What are the sources of error?
	\end{itemize}
	\item Conclusion
\end{itemize}
%
\subsubsection*{Error Handling}
All data in your diagrams should be displayed with both x and y error bars.
When carrying out the experiment, you will already receive the standard deviations for the measured values from the Gaussian fits. 
You can make sensible assumptions about the uncertainties of the other measured values. 
Create tables with measured values and associated uncertainties and explicitly state the formulas used for the calculation.
For variables that are calculated from several (error-prone) measured values, carry out Gaussian error propagation:
\begin{equation}\label{eq:RSS}
	\sigma_{f} = \sqrt{\left( \frac{\partial f}{\partial x_1}\ \sigma_{x_1} \right)^2 + \left( \frac{\partial f}{\partial x_2}\ \sigma_{x_2} \right)^2 + ... + \left( \frac{\partial f}{\partial x_n}\ \sigma_{x_n} \right)^2} 
\end{equation}
Here, $\sigma_{f}$ is the uncertainty of the quantity $f = f (x_1, x_2, ... , x_n)$, which is dependent of the measured values $x_1, x_2, ... , x_n$ and their respective uncertainties $\sigma_{x_1}, \sigma_{x_2}, ..., \sigma_{x_n}$. 

Alternatively, estimate the maximum error limits with
\begin{equation}\label{eq:abs}
		\sigma_{f} = \left| \frac{\partial f}{\partial x1}\right| \sigma_{x1} + \left| \frac{\partial f}{\partial x2}\right| \sigma_{x2} + ... + \left| \frac{\partial f}{\partial xn}\right| \sigma_{xn}
\end{equation}
%
\subsection{Notes}
%
\subsubsection*{Weighted Mean}
The weighted mean $\overline{x}$ of several measured values $x_i$ of the same quantity $x$ is defined as
\begin{subequations}\label{eq:weighted_mean}
    \begin{align}
    \overline{x} & = \frac{\sum\limits_{i}\ w_i \ x_{i}}{\sum\limits_{i} \ w_i }  \label{eq:wmean}\\
    \text{with weights } \quad w_i & = (\Delta x_{i})^{-2} \\
    \text{and error } \quad \Delta \overline{x} & = \left(  \sum\limits_{i} w_i  \right)^{-\frac{1}{2}} \label{eq:wmeanerr}
    \end{align}
\end{subequations}
here, $\Delta x_i$ is the uncertainty of the $i$-th single measurement value.