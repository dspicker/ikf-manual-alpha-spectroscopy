\newpage
\appendix
%
\section{Anhang}\label{sec:appendix}
%
\subsection{Das Protokoll}
Reichen Sie zusammen mit dem Protokoll Ihre Original-Messdaten als ordentliche \texttt{.txt}-Datei ein, mit entsprechenden Kommentaren, um die Daten den jeweiligen Messungen zuordnen zu können. Der Aufbau des Protokolls sollte sich an den folgenden Punkten orientieren: 
%
\begin{itemize}
	\item Theorieteil
	\item Durchführung und Auswertung
	\begin{itemize}
		\item Was soll gemessen werden und warum?
		\item Welche Messergebnisse werden erwartet?
		\item Wie wird die Messung durchgeführt?
		\item Was wurde tatsächlich gemessen? Wie groß sind die Messfehler?
		\item Was kann aus dem Messergebnis gelernt werden? Wurden die Erwartungen erfüllt? Was sind die Fehlerquellen?
	\end{itemize}
	\item Fazit
\end{itemize}
%
\subsubsection*{Fehlerbehandlung}
Alle Daten in Ihren Diagrammen sollten sowohl mit x- als auch mit y-Fehlerbalken dargestellt werden.
Sie erhalten bei der Durchführung des Experimentes aus den Gauß-Fits bereits die Standardabweichungen zu den Messwerten. 
Für die übrigen Messwerte können Sie jeweils sinnvolle Annahmen zur Unsicherheit machen. 
Erstellen Sie Tabellen mit Messwerten und zugehörigen Unsicherheiten und weisen Sie die zur Berechnung verwendeten Formeln explizit aus.
Bei Größen, die aus mehreren (fehlerbehafteten) Messwerten ausgerechnet werden, führen Sie Gaußsche Fehlerfortpflanzung durch:
\begin{equation}\label{eq:RSS}
	\sigma_{f} = \sqrt{\left( \frac{\partial f}{\partial x_1}\ \sigma_{x_1} \right)^2 + \left( \frac{\partial f}{\partial x_2}\ \sigma_{x_2} \right)^2 + ... + \left( \frac{\partial f}{\partial x_n}\ \sigma_{x_n} \right)^2} 
\end{equation}
Hierbei errechnet sich die Messunsicherheit $\sigma_{f}$ von der Größe $f = f (x_1, x_2, ... , x_n)$, die abhängig ist von den Messwerten $x_1, x_2, ... , x_n$ und deren Unsicherheiten $\sigma_{x_1}, \sigma_{x_2}, ..., \sigma_{x_n}$. 
%
\subsection{Anmerkungen}
%
\paragraph{Der gewichtete Mittelwert}
$\overline{x}$ von mehreren Messwerten $x_i$ einer Messgröße $x$ ist definiert als
\begin{subequations}\label{eq:weighted_mean}
    \begin{align}
    \overline{x} & = \frac{\sum\limits_{i}\ w_i \ x_{i}}{\sum\limits_{i} \ w_i }  \label{eq:wmean}\\
    \text{mit Gewichten } \quad w_i & = (\Delta x_{i})^{-2} \\
    \text{und Unsicherheit } \quad \Delta \overline{x} & = \left(  \sum\limits_{i} w_i  \right)^{-\frac{1}{2}} \label{eq:wmeanerr}
    \end{align}
\end{subequations}
hierbei ist $\Delta x_i$ die Unsicherheit der $i$-ten Einzelmessung.